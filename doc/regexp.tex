\documentclass[a4paper,francais]{article}
\usepackage[T1]{fontenc}
\usepackage[latin1]{inputenc}
\begin{document}

les expressions comprises par Eliot ne sont qu'un sous ensemble des
expressions rationnelles habituelles.

\section{utilisation}

Les mots recherch�s sont complets : la recherche d'une expression
\verb=e= correspond � l'expression \verb=^e$=. Pour rechercher un 
motif \verb=m= dans un mot il faut donc utiliser l'expression \verb=.*m.*=

\subsection{caract�res}

\begin{itemize}
\item \texttt{a} � \texttt{z} : le caract�re indiqu�
\item \texttt{.} :  n'importe quel caract�re
\item \texttt{:v:} : n'importe quelle voyelle 
\item \texttt{:c:} : n'importe quelle consonne
\item \texttt{:1:} : liste 1 d�finie par l'utilisateur
\item \texttt{:2:} : liste 2 d�finie par l'utilisateur
\end{itemize}

\subsection{r�p�titions}

\begin{itemize}
\item \texttt{+} une fois ou plus
\item \texttt{*} z�ro fois ou plus
\end{itemize}

\subsection{disjonction}

\begin{itemize}
\item \texttt{[abc]} : \texttt{a} ou \texttt{b} ou \texttt{c}
\end{itemize}

\subsection{groupement}

\begin{itemize}
\item \texttt{(abc)} : la cha�ne \texttt{abc}
\end{itemize}

\subsection{exemples}

\begin{itemize}
\item \verb=a.*= : liste des mots d�butant par la lettre \verb=a=
\item \verb=.*a= : liste des mots se terminant par la lettre \verb=a=
\item \verb=.*oula.*= : liste des mots contenant le motif \verb=oula=
\item \verb=a.*b= : liste des mots d�butant par \verb=a= et se terminant par \verb=b=
\item \verb=.*a.*e.*i.*o.*u.*= : liste des mots contenant les les lettres \verb=aeiou= dans l'ordre
\end{itemize}

\end{document}
